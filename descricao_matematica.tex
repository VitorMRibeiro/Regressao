\documentclass{article}

\usepackage{amsmath}

\begin{document}

Notação

Negrito = vetor.

Exemplos:

\(\textbf{x}\) é um vetor \([x_0,...,x_n]^T\)

\(\boldsymbol{\phi}(x)\) é um vetor de funções \([\phi_0(x),...,\phi_n(x)]^T\)

\section{Introdução}

Este texto é um resumo de alguns conceitos sobre regressão apresentados no livro \emph{Pattern Recognition and Machine Learning},
e a documentação de algumas ideias que eu tive sobre como gerar curvas de regressão suaves minimizando, além
do erro entre as predições e os dados, a derivada segunda da curva de regressão. O texto serve a quem
desejar compreender os algorítimos implementados aqui, e a mim como referência futura.

\section{Regressão}

O objetivo da regressão é fazer a previsões dos valores de uma \emph{target variable} \(t\) dado o valor de uma
\emph{input variable} \(\bf x\). No geral \(\bf x\) é um vetor de dimensão qualquer, mas tratarei aqui apenas do
caso em que \( \bf x \) tem dimensão 1.

Dado um conjunto de dados composto de observações \( \{ \textbf{x}_n \} \) e suas respectivas
\emph{target variables} \( \{t_n\} \), o que se pretende fazer é a previsão do valor de \(t\)
dado um valor de \(\bf x\) que não necessariamente está presente nas observações. Em outras palavras,
quer-se encontrar uma função \( y(\textbf{x}) \) que associa cada valor possível de \(\bf x\) a uma previsão de
\(t\).

A função \( y(\textbf{x}) \) pode ser encontrada através do ajuste de parâmetros de uma função
\( y(\textbf{x}, \textbf{w}) \), onde \( \textbf{w} \) é um vetor de parâmetros ajustáveis. Os valores
finais dos parâmetros ajustáveis \( \textbf{w}_f \) devem ser escolhidos de maneira a minimizar
(ou quase isso) o valor de \( \mathcal{L}(\textbf{w}) \), onde \( \mathcal{L} \) é uma função de perda
adequada. A função de perda \( \mathcal{L} \) é o criterio pelo qual se julga o quão adequadas são as
previsões de \( y(\textbf{x}) = y(\textbf{x}, \textbf{w}_f) \) dado o conjunto de dados.

Um método pelo qual os parâmetros ajustáveis podem ser atualizados é o da \emph{descida de gradiente}, onde os parâmetros
são modificados de maneira iterativa na direção do negativo do gradiente da função de perda,
\[ \textbf{w}_{n + 1} = \textbf{w}_n - \lambda \nabla \mathcal{L}(\textbf{w}) \]
ou seja, a cada passo os parâmetros ajustáveis \( \textbf{w} \) são modificados de maneira a diminuir o valor da
função de perda \( \mathcal{L}(\textbf{w}) \). Em alguns casos, porém, uma fórmula fechada para os valores ótimos de \( \textbf{w} \)
pode ser encontrada, como será visto mais adiante.

\section{Modelos Lineares de Regressão}

Os modelos lineares de regressão são aqueles cuja função \( y(\textbf{x}, \textbf{w}) \) é linear em relação aos parâmetros
ajustáveis \( \textbf{w} \), o que não quer dizer que \( y(\textbf{x}, \textbf{w}) \) seja necessariamente linear com relação
a \( \textbf{x} \). No geral, \( y(\textbf{x}, \textbf{w}) \) é uma combinação linear de \emph{basis functions}
\( \phi(\textbf{x}) \), funções de \( \textbf{x} \) que podem ou não ser lineares,
\[ y(\textbf{x}, \textbf{w}) = w_0 + \sum_{i=1}^{M-1} w_i \phi_i(\textbf{x}_n) \]
onde M é a quantidade de parâmetros ajustáveis. A equação anterior pode ser simplificada para \( \textbf{w}^T \boldsymbol{\phi}(\textbf{x}) \)
usando notação vetorial, assumindo que \( \phi_0 = 1 \).

A função de perda utilizada aqui é a soma dos quadrados dos erros, isto é, as diferenças entre as previsões do modelo e o
valor real das observações \( t_n - y(\textbf{x}_n, \textbf{w}) \) são elevadas ao quadrado e somadas, por todas as instâncias
de observação, para gerar um valor que representa o quão adequada é a escolha de valores para \(\textbf{w}\).
\[ \mathcal{L}(\textbf{w}) = \frac{1}{2} \sum_{n=1}^{N} \{ t_n - y(\textbf{x}_n, \textbf{w})\}^2 \]
onde \(N\) é a quantidade de observações. Tal escolha de função de perda é justificada se assumirmos que os dados observados
são gerados a partir de uma função determinística somada a um ruído gaussiano, o que é o caso, os dados artificiais gerados aqui
são da forma \( t = \mathcal{N}(t | sin(x), \sigma) \).

Essa escolha de função de perda, combinada ao fato de \( y(\textbf{x}, \textbf{w}) = \textbf{w}^T \boldsymbol{\phi}(\textbf{x}) \) ser linear
em relação a \( \textbf{w} \), nos leva a seguinte equação para o gradiente da função de perda
\[ \nabla \mathcal{L}(\textbf{w}) = \sum_{n=1}^{N} \{ t_n - \textbf{w}^T \boldsymbol{\phi}(\textbf{x}_n) \} \boldsymbol{\phi}(\textbf{x}_n)^T \]

Seja
\[ \boldsymbol{\Phi} = \begin{bmatrix}
        \phi_0(x_0) & \phi_1(x_0) & \dotsb & \phi_{M-1}(x_0) \\
        \phi_0(x_1) & \phi_1(x_1) & \dotsb & \phi_{M-1}(x_1) \\
        \vdots & \vdots & \ddots & \vdots \\
        \phi_0(x_n) & \phi_1(x_n) & \dotsb & \phi_{M-1}(x_n)
    \end{bmatrix},
\]
a equação anterior para o gradiente da função de perda pode ser reescrita da seguinte maneira:
\[
    \nabla \mathcal{L}(\textbf{w}) = (\textbf{t}^T - \textbf{w}^T \boldsymbol{\Phi}^T) \boldsymbol{\Phi}
\]
expandindo as matrizes,
\[
    \nabla \mathcal{L}(\textbf{w}) =
    \begin{bmatrix} t_0 - \textbf{w}^T \boldsymbol{\phi}(x_0), \dotsb, t_n - \textbf{w}^T \boldsymbol{\phi}(x_n)\end{bmatrix}
    \begin{bmatrix}
        \phi_0(x_0) & \dotsb & \phi_{M-1}(x_0) \\
        \vdots & \ddots & \vdots \\
        \phi_0(x_n) & \dotsb & \phi_{M-1}(x_n)
\end{bmatrix}
\]
fazendo o gradiente da função de perda igual a zero temos,
\[
    (\textbf{t}^T - \textbf{w}^T \boldsymbol{\Phi}^T) \boldsymbol{\Phi} = 0
\]
\[
    \textbf{t}^T \boldsymbol{\Phi} - \textbf{w}^T \boldsymbol{\Phi}^T \boldsymbol{\Phi} = 0
\]
\[
    \textbf{w}^T \boldsymbol{\Phi}^T \boldsymbol{\Phi} = \textbf{t}^T \boldsymbol{\Phi}
\]
\[
    \boldsymbol{\Phi}^T \boldsymbol{\Phi} \textbf{w} = \boldsymbol{\Phi}^T \textbf{t}
\]
\[
    \textbf{w} = (\boldsymbol{\Phi}^T \boldsymbol{\Phi})^{-1} \boldsymbol{\Phi}^T \textbf{t}
\]
uma fórmula fechada para os valores ótimos de \( \textbf{w} \).

\section{Regularização}

Se o modelo sendo ajustado é flexível demais para a quantidade de dados disponíveis, o resultado
é uma curva cheia de mudanças bruscas que falha em capturar o padrão principal presente nos dados, pois,
sendo flexível demais, se adapta ao ruído nos dados, o que faz ofuscar o padrão geral. Esse problema é
denominado \emph{overfitting}.

Uma maneira de lidar com o problema de \emph{overfitting} é a penalização de valores muito grandes para
os parâmetros \( \textbf{w} \). Uma maneira simples de fazer isso é adicionar a soma dos quadrados dos
parametros \( \textbf{w}^T \textbf{w} \) a função de perda
\[ \mathcal{L}(\textbf{w}) = \frac{1}{2} \sum_{n=1}^{N} \{ t_n - y(\textbf{x}_n, \textbf{w})\}^2 + \frac{1}{2} \lambda \textbf{w}^T \textbf{w} \]
onde \( \lambda \) é o coeficiente de regularização, que controla a força da penalização aplicada.

Da adição do fator do termo de regularização na função de perda, resulta o seguinte gradiente
\[
    \nabla \mathcal{L}(\textbf{w}) = (\textbf{t}^T - \textbf{w}^T \boldsymbol{\Phi}^T) \boldsymbol{\Phi} + \lambda \textbf{w}^T
\]
A vantagem de se utilizar a soma dos quadrados como termo de regularização é que a função de perda resultante
ainda é uma função quadrática do vetor de parâmetros \( \textbf{w} \), portanto, possui formula fechada para
os valores ótimos de \( \textbf{w} \)
\begin{gather}
    \nonumber
    (\textbf{t}^T - \textbf{w}^T \boldsymbol{\Phi}^T) \boldsymbol{\Phi} + \lambda \textbf{w}^T = 0              \\ \nonumber
    \textbf{w}^T \boldsymbol{\Phi}^T \boldsymbol{\Phi} - \lambda \textbf{w}^T = \textbf{t}^T \boldsymbol{\Phi}  \\
    \boldsymbol{\Phi}^T \boldsymbol{\Phi} \textbf{w} - \lambda \textbf{w} = \boldsymbol{\Phi}^T \textbf{t}      \\ \nonumber
    \boldsymbol{\Phi}^T \boldsymbol{\Phi} \textbf{w} - \lambda I_M \textbf{w} = \boldsymbol{\Phi}^T \textbf{t}  \\ \nonumber
    (\boldsymbol{\Phi}^T \boldsymbol{\Phi} - \lambda I_M) \textbf{w} = \boldsymbol{\Phi}^T \textbf{t}
\end{gather}
\begin{equation}
    \textbf{w} = (\boldsymbol{\Phi}^T \boldsymbol{\Phi} - \lambda I_M)^{-1} \boldsymbol{\Phi}^T \textbf{t}
\end{equation}

A equação (2) contém um erro de sinal, por algum motivo o resultado que segue do desenvolvimento em (1) tem o termo
\( - \lambda I_M \) enquanto no livro esse termo é positivo. Não consegui entender de onde surgiu esse erro, mas não tem muita
importância, visto que o termo já esta sendo multiplicado por um fator \( \lambda \).

\section{Penalização de curvatura}

Me veio a ideia de que as curvas de regressão poderiam ser feitas mais suaves se minimizarmos, além do erro, a derivada
de segunda ordem da curva \( y(\textbf{x}, \textbf{w}) = \textbf{w}^T \boldsymbol{\phi}(\textbf{x}) \). Para tal, adiciona-se
o seguinte termo a função de perda
\begin{equation}
    \int_{a}^{b} \bigg( \frac{d^2}{dx^2} [ \textbf{w}^T \boldsymbol{\phi}(x) ] \bigg)^2 dx
\end{equation}
o intervalo \( [a, b] \) é o intervalo relevante para a regressão.

A rigor, o termo em (3) não representa corretamente a curvatura de \( y(\textbf{x}, \textbf{w}) \), a representação
correta utilizaria o módulo da derivada segunda de \( y \) a fim de penalizar tanto as curvas convexas quanto as concavas, ao
invés do quadrado; o quadrado é mais coveniente, no entanto.

Da adição de termo (3) na função de perda resulta
\begin{equation}
    \nabla \mathcal{L}(\textbf{w}) = (\textbf{t}^T - \textbf{w}^T \boldsymbol{\Phi}^T) \boldsymbol{\Phi} + \lambda \textbf{w}^T \boldsymbol{\Phi}''
\end{equation}
onde
\[
\boldsymbol{\Phi}'' =
    \begin{bmatrix}
        \int \phi''_0(x) \phi''_0(x) dx & \int \phi''_0(x) \phi''_1(x) dx & \dotsb & \int \phi''_0(x) \phi''_n(x) dx \\
        \int \phi''_1(x) \phi''_0(x) dx & \int \phi''_1(x) \phi''_1(x) dx & \dotsb & \int \phi''_1(x) \phi''_n(x) dx \\
        \vdots & \vdots & \ddots & \vdots \\
        \int \phi''_n(x) \phi''_0(x) dx & \int \phi''_n(x) \phi''_1(x) dx & \dotsb & \int \phi''_n(x) \phi''_n(x) dx
    \end{bmatrix}
\]

De maneira similar aos casos anteriores, uma fórmula fechada para \( \textbf{w} \) pode ser encontrada
\[
    \textbf{w} = (\boldsymbol{\Phi}^T \boldsymbol{\Phi} - \lambda \boldsymbol{\Phi}'' )^{-1} \boldsymbol{\Phi}^T \textbf{t}
\]
como em (2) essa equação também provavelmente tem um erro de sinal.

A título de exemplo, quando as bases são polinômios \( \phi_i(x) = x^i \) os elementos \( \phi''_{i j} \) de \( \boldsymbol{\Phi}'' \)
são
\[
    \phi''_{i j} = \Bigg(\frac{(j^2 - j)(i^2 - i)x^{i+j-3}}{i+j-3} \Bigg) \biggr\rvert_{a}^{b}
\]

\section{Referências}

\emph{Pattern Recognition and Machine Learning - Christopher Bishop, Capítulo 3.}
\end{document}